\documentclass{article}
\usepackage{amsmath}
\usepackage{amssymb}
\begin{document}
\section{Aussagenlogik}
\subsection{Signatur $\Sigma$}
Eine (aussagenlogische) \textit{Signatur} ist eine abz\"ahlbare Menge $\Sigma$
von Symbolen, etwa
\begin{align*}
    \Sigma &= \{P_0 , \dots, P_n \} \\ 
    \text{oder } \Sigma &= \{P_0 , P_1, \dots\}
\end{align*}
Die Elemente von $\Sigma$ hei{\ss}en auch \textit{atomare Aussagen}, \textit{Atome}
oder \textit{Aussagevariablen}.

\subsection{Formeln $For0_\Sigma$}
$For0_\Sigma$ ist die Menge der \textit{Formeln} \"uber $\Sigma$
induktiv definiert durch
\[ 1 \in For0_\Sigma, 0 \in For0_\Sigma, \Sigma \subseteq For0_\Sigma \]

Wenn $A, B \in For0_\Sigma$, dann sind auch
\[ \lnot A, (A \land B), (A \lor B), (A \rightarrow B), (A \leftrightarrow B) \]
Elemente von $For0_\Sigma$

\subsection{Interpretation $I$}
Es sei $\Sigma$ eine aussagenlogische Signatur. Eine \textit{Interpretation}
\"uber $\Sigma$ ist eine beliebige Abbildung
\[ I : \Sigma \rightarrow \{W, F \} \]

\subsection{Auswertung $val_I$}
Zu jeder Interpretation $I$ \"uber $\Sigma$ wird eine zugeh\"orige \textit{Auswertung} der Formeln \"uber $\Sigma$ definiert
\[ val_I : For0_\Sigma \rightarrow {W, F} \]
mit:
\begin{align*}
    val_I(1) &= W \\
    val_I(0) &= F \\
    val_I(P) &= I(P) \qquad \text{f\"ur jedes } P \in \Sigma \\
    val_I(\lnot A) &=  \begin{cases}
        F \text{ falls } &val_I(A) = W \\
        W \text{ falls } &val_I(A) = F \\
    \end{cases}
\end{align*}

\subsection{Logische Grundbegriffe}
\begin{description}
    \item[Modell] Ein \textit{Modell} einer Formel $A \in For0_\Sigma$ ist eine Interpretation $I$ \"uber $\Sigma$ mit $val_I(A) = W$.

    ( Eine Belegung $I$ der Aussagenvariablen die die Formel $A$ wahr macht ist ein Modell.)

\item[Modell - Formelmenge] Zu einer \textit{Formelmenge} $M \subseteq For0_\Sigma$ ist ein Modell von $M$ eine Interpretation I, welche Modell von jedem $A \in M$ ist. 
    
    (Ein Modell - mit einer Interpretation $I$ - einer Formelmenge ist f\"ur jede Formel in der Menge ein Modell.)

\item[Allgemeing\"ultigkeit]
    $A \in For0_\Sigma$ hei{\ss}t \textit{allgemeing\"ultig} gdw. 
    $val_I(A) = W$ f\"ur jede Interpretation I \"uber $\Sigma$.

    (Eine Formel ist allgemein\"ultig wenn sie unter jeder Interpretation wahr ist.)

\item[Erf\"ulbar]
    $A \in For0_\Sigma$ hei{\ss}t \textit{erf\"ullbar} gdw.
    es gibt eine Interpretation $I$ \"uber $\Sigma$ mit $val_I(A) = W$.

    (Eine Formel ist erf\"ullbar wenn sie unter einer Interpretation wahr ist.)

\item[Folgerung $\models$]
    Sei $\Sigma$ eine Signatur, $M \subseteq For0_\Sigma, A \in For0_\Sigma$ \\
    $M \models A$ lies: \textit{aus M folgt A} gdw.
    Jedes Modell von $M$ ist auch Modell von $A$.

    (Jede Interpretation von $M$ ist auch eine Interpretation von $A$)

\item[Logische \"Aquivalenz]
    $A, B \in For0_\Sigma$ hei{\ss}en \textit{logisch \"aquivalent} gdw.
    \[ \{ A\} \models_\Sigma B \text{ und } \{B\} \models_\Sigma A \]
\end{description}
\subsection{Shannon-Formeln}

\subsection{Horn-Formeln}
Eine aussagenlogische Formel $A$ ist eine \textit{Horn-Formel}, wenn
\begin{itemize}
    \item $A$ in KNF ist
    \item jede Disjunktion in $A$ h\"ochstens ein positives Literal enth\"alt
\end{itemize}
Alternative Schreibweisen:
\begin{align*}
    \lnot P_1 \lor \cdots \lor \lnot P_n \lor A &\equiv
P_1 \land \cdots \land P_n \rightarrow A \\
    \lnot P_1 \lor \cdots \lor \lnot P_n &\equiv
P_1 \land \cdots \land P_n \rightarrow 0
\end{align*}
Bezeichnungen: $\lnot P_1 \lor \cdots \lor \lnot P_n$: \textit{Rumpf},
\qquad $A$: \textit{Kopf} (bei leerem Rumpf: \textit{Fakt})

\paragraph{Erf\"ullbarkeit} 
F\"ur Horn-Formeln ist die Erf\"ullbarkeit in 
$O(n^2)$ entscheidbar.

\subsection{Davis-Putnam-Logemann-Loveland (DPLL) Verfahren}
Wichtigstes Verfahren zur Entscheidung des allgemeinen Erf\"ullbarkeitsproblems (SAT-Problem). Eingabe in KNF. Worst case complexity $O(2^n)$.


\subsection{Schreibweisen}
$\Box$ f\"ur die leere Klausel \\
$\emptyset$ f\"ur die leere Klauselmenge \\
Es gilt: $I(\emptyset) = W$ \\
Es gilt: $I(\Box) = F$ \\

\section{Pr\"adikatenlogik erster Stufe}
\subsection{Logische Zeichen}
\paragraph{Wie in der Aussagenlogik:}
$\lnot, \land, \lor, \rightarrow, \leftrightarrow, (, )$
\paragraph{Neu:} 
\begin{align*}
    \forall \qquad &\text{Allquantor} \\
    \exists  \qquad &\text{Existenzquantor} \\
    v_i  \qquad &\text{Individuenvariablen, } i \in \mathbb{N} \\
    \doteq  \qquad &\text{objektsprachliches Gleichheitssymbol} \\
    ,  \qquad &\text{Komma} \\
\end{align*}
Mit \textit{Var} bezeichnet man die zur Verf\"ugung stehenden Variablen.

\subsection{Signatur}
Eine \textit{Signatur} ist ein Tripel $\Sigma = ( F_\Sigma, P_\Sigma,
\alpha_\Sigma)$ mit:
\begin{itemize}
    \item $F_\Sigma, P_\Sigma$ sind endliche oder
        abz\"ahlbar unendliche Mengen
    \item $F_\Sigma, P_\Sigma$ und die Menge der Sondersymbole
        sind paarweise disjunkt
    \item $\alpha_\Sigma: F_\Sigma \cup P_\Sigma \rightarrow \mathbb{N}$.
\end{itemize}
$f \in F_\Sigma$ hei{\ss}t \textit{Funktionssymbol}, 
$p \in P_\Sigma$ hei{\ss}t \textit{Pr\"adikatssymbol}.\\
f ist \textit{n-stelliges Funktionssymbol}, wenn $\alpha_\Sigma(f) = n$; \\
p ist \textit{n-stelliges Pr\"adikatssymbol}, wenn $\alpha_\Sigma(p) = n$; \\
Ein nullstelliges Funktionssymbol hei{\ss}t auch \textit{Konstantensymbol}
oder kurz \textit{Konstante}, \\
ein nullsteliges Pr\"adikatssymbol ist ein \textit{aussagenlogisches Atom}.

\subsection{Terme}
$Term_\Sigma$, die Menge der \textit{Terme \"uber $\Sigma$}, ist induktiv definiert durch
\begin{enumerate}
    \item $Var \subseteq Term_\Sigma$
    \item Mit $f \in F_\Sigma, $ \\
        $\alpha_\Sigma(f) = n, \\
        t_1, \cdots, t_n \in Term_\Sigma$ \\
        ist auch $f(t_1, \cdots, t_n) \in Term_\Sigma$
\end{enumerate}
Ein Term hei{\ss}t \textit{Grundterm}, wenn er keine Variablen enth\"alt.

\subsection{Formeln}
$At_\Sigma$ ist die Menge der \textit{atomaren Formeln} \"uber $\Sigma$:
\begin{align*}
    At_\Sigma := & \{s \doteq t | s,t \in Term_\Sigma \} \cup \\
    & \{ p(t_1, \cdots t_n) | p \in P_\Sigma, \alpha_\Sigma(p) = n,
    t_i \in Term_\Sigma \}
\end{align*}
\\
$For_\Sigma$, die Menge der \textit{Formeln \"uber $\Sigma$}, ist 
induktiv definiert durch
\begin{enumerate}
    \item ${1, 0} \cup At_\Sigma \subseteq For_\Sigma$
    \item Mit $x \in Var$ und $A, B \in For_\Sigma$ sind ebenfalls in 
        $For_\Sigma$: 
        \[ \lnot A, (A \land B), (A \lor B), (A \rightarrow B),
        (A \leftrightarrow B), \forall xA, \exists xA \]
\end{enumerate}

\paragraph{Substitution}
Eine \textit{Substitution} ist eine Abbildung
\[ \sigma : Var \rightarrow Term_\Sigma \]
mit $\sigma(x) = x$ f\"ur fast alle $x \in Var$. \\
$\sigma$ hei{\ss}t \textit{Grundsubstitution}, wenn f\"ur alle $x$ mit
$\sigma(x) \neq x$ der Funktionswert $\sigma(x)$ ein Grundterm ist. \\
\paragraph{kollisionsfreie Substitution}
Eine Substitution $\sigma$ hei{\ss}t \textit{kollisionsfrei} f\"ur eine Formel $A$,
wenn f\"ur jede Variable $z$ und jede Stelle freien Auftretens von $z$ in $A$ gilt: \\
Diese Stelle liegt nicht im Wirkungsbereich eines Pr\"afixes $\forall x$ oder
$\exists x$, wo $x$ eine Variable in $\sigma(z)$ ist.
\[ \mu_1 = \{x/y\} \text{ist nicht kollisionsfrei f\"ur }
\forall y p(x, y) \]

\subsection{Interpretation}
Es sei $\Sigma$ eine Signatur der PL1. \\
Eine \textit{Interpretation} $D$ von $\Sigma$ ist ein Paar $(D, I)$ mit
\begin{enumerate}
    \item $D$ ist eine beliebige, nichtleere Menge
    \item $I$ ist eine Abbildung der Signatursymbole, die
        \begin{itemize}
            \item jeder Konstanten $c$ ein Elemente $I(c) \in D$
            \item f\"ur $n \geq 1$: jedem n-stelligen Funktionssymbol $f$
                eine Funktion $I(f): D^n \rightarrow D$
            \item jedem 0-stelligen Pr\"adikatssymbol P ein Wahrheitswert
                $I(P) \in \{W, F\}$
            \item f\"ur $n \geq 1$: jedem n-stelligen Pr\"adikatssymbol
                $p$ eine n-stellige Relation $I(p) \subseteq D^n$ zuordnet.
        \end{itemize}
\end{enumerate}

\subsection{Variablenbelegung}
Es sei $(D, I)$ eine Interpretation von $\Sigma$. \\
Eine \textit{Variablenbelegung} (oder kurz \textit{Belegung} \"uber $D$)
ist eine Funktion
\[ \beta: Var \rightarrow D. \]
Zu $\beta, x \in Var$ und $d \in D$ definieren wir die \textit{Modifikation} von $\beta$ and der Stelle $x$ zu $d$:
\begin{align*}
    \beta_x^d(y) = \begin{cases}
        d &\text{ falls } y = x \\
        \beta(y) &\text{ falls } y \neq x
    \end{cases}
\end{align*}

\subsection{Auswertung}
\paragraph {Auswertung von Termen}
Sei $(D, I)$ Interpretation von $\Sigma, \beta$ Variablenbelegung \"uber D.
Wir definieren eine Funktion $val_{D,I,\beta}$, mit
\begin{align*}
    val_{D,I,\beta}(t) \in D & \text{ f\"ur } t \in Term_\Sigma \\
    val_{D,I,\beta}(A) \in \{W, F\} & \text{ f\"ur } A \in For_\Sigma
\end{align*}

\paragraph{Auswertung von Formeln}
\begin{align*}
val_{D,I,\beta} (1) &= W \\
val_{D,I,\beta} (0) &= F \\
val_{D,I,\beta}(s\doteq t) &:= \begin{cases}
    W &\text{ falls } val_{D,I,\beta}(s) = val_{D,I,\beta}(t) \\
    F &\text{ sonst}
\end{cases} \\
val_{D,I,\beta}(P) &:= I(P) \text{ f\"ur 0-stellige Pr\"adikate } P \\
val_{D,I,\beta}(p(t_1, \cdots t_n)) &:= \begin{cases}
    W &\text{ falls } (val_{D,I,\beta}(t_1), \cdots, val_{D,I,\beta}(t_n))
    \in I(p) \\
    F &\text{ sonst }
\end{cases} \\
val_{D,I,\beta} (X) & \text{ f\"ur } X \in \{\lnot A, A \land B, A \lor B, 
A \rightarrow B, A \leftrightarrow B \} \text{ wie in der 
Aussagenlogik.} \\
val_{D,I,\beta}(\forall x A) &:= \begin{cases}
    W & \text{ falls f\"ur alle } d \in D : val_{D,I,\beta_x^d}(A) = W \\
    F & \text { sonst}
\end{cases} \\
val_{D,I,\beta}(\exists x A) &:= \begin{cases}
    W & \text{ falls ein } d \in D \text{ existiert mit } val_{D,I,\beta_x^d}(A) = W \\
    F & \text { sonst}
\end{cases}
\end{align*}
\subsection{Unifikation}
Es sei $T \subseteq Term_\Sigma, T \neq \{\}$, und $\sigma$ eine Substitution
\"uber $\Sigma$. \\
$\sigma$ \textit{unifiziert} $T$, oder: \\
$\sigma$ ist \textit{Unifikator von T},
genau dann, wenn $\#\sigma(T) = 1$. \\
T hei{\ss}t \textit{unifizierbar}, wenn T einen Unifikator besitzt. \\
Insbesondere sagen wir f\"ur zwei Terme $s, t$ dass $s$ \textit{unifizierbar}
sei \textit{mit} $t$, wenn $\sigma(t) = \sigma(s)$.

\paragraph{Allgemeinster Unifikator} 
Es sei $T \subseteq Term_\Sigma$. \\
Ein \textit{allgemeinster Unifikator} oder mgu (\textit{most general unifier}) von T ist eine Substitution $\mu$ mit
\begin{enumerate}
        \item $\mu$ unifiziert $T$
    \item Zu jedem Unifikator $\sigma$ von $T$ gibt es eine Substitution
        $\sigma'$ mit $\sigma = \sigma' \circ \mu$.
\end{enumerate}
\end{document}
